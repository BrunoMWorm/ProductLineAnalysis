\documentclass[11pt, english]{article}
\usepackage{babel}

%% macros
% reference macros
%\newcommand{\Classen}{\cite{Classen:2010:MCL:1806799.1806850}}
\begin{document}

\title{A Taxonomy of Software Product Line Analyses}

\author{Ramy Shahin \\
  University of Toronto \\
 ramy.shahin@cs.toronto.edu
 \and
  Marsha Chechik \\
  University of Toronto \\
 marsha.chechik@cs.toronto.edu}

\maketitle

\begin{abstract}
TODO
\end{abstract}

\section{Introduction}
TODO

\section{Motivation}
Product line analyses have been surveyed and classified according to different criteria (e.g., \cite{Thum:2014}). 
Modularity of analyses directly impacts the degree of variability the analysis needs to deal with. Since presence conditions are propositional formulas, propositional reasoning is an essential part of any Product Line analysis. The number of propositional variables (which reflects the degree of variability) in formulas is an important factor when it comes to picking the propositional reasoning component used in the analysis. For example, with hundreds or thousands of variables, SAT solvers are usually used. On the other hand, if the number of variables is much lower (10-20), Binary Decision Diagrams (BDDs) tend to perform much better. The difference between both reasoning techniques was demonstrated in the difference in performance between Typechef \cite{Kastner:2011} and SuperC \cite{Gazzillo:2012}.

\subsection{Modularity}

Modularity is a property of product-level analyses. An analysis can be either modular or monolithic. 
\begin{description}
\item [Modular Analysis]
is applied to units (components) of a system rather than the system as a whole. The results of unit-by-unit analysis are then aggregated to provide the analysis result for the whole system.
\end{description}

The granularity of that component varies from one analysis to another. In many programming languages, type checkers work at the level of a compilation unit or a module. Intra-procedural analyses on the other hand work at the level of a single procedure. 

\begin{description}
\item [Monolithic Analysis]
is applied to the system as a whole. 
\end{description}

Examples of monolithic analyses include monolithic model checkers that build a state machine of the whole model with no decomposition into smaller units.

\subsection{Derivability}

In the context of this paper derivability is an attribute of Product Line analyses.

\begin{description}
\item [Derived Analysis]
is a  Product Line analysis that is built on top of its corresponding product-level analysis. 
\end{description}

An analysis can be derived in one of two ways:

\begin{itemize}

\item A product-level analysis can be used as a black box. In such cases the Product Line analysis calls into the product level analysis and aggregates the results for the whole Product Line.

\item A product-level analysis can be used as a white box. Some automatic technique is used to lift the implementation of the product-level analysis into a Product Line analysis. Since this process is automated, we still consider this approach modular.
\end{itemize}

\begin{description}

\item [Rewritten Analysis]
is a Product Line analysis written from scratch independently of its corresponding product level analysis.
\end{description}

\section{Taxonomy}

\begin{table*}[t]
\begin{center}
\begin{tabular}{| l | l | l |}
\hline
	& Modular & Monolithic \\ \hline
	Derived & 
		SPL\textsuperscript{LIFT} \cite{Bodden:2013} & 
		FTS \cite{Classen:2010} \\ \hline
	Rewritten & 
		\begin{tabular}{@{}c@{}}
			TypeChef \cite{Kastner:2011} \\ 
			SuperC \cite{Gazzillo:2012}
		\end{tabular} &
			Model Transformations \cite{Salay:2014} \\ \hline
\end{tabular}
\end{center}
\caption{A taxonomy of Software Product Line analyses}
\label{table:taxonomy}
\end{table*}

\subsection{Modular and Derived}

\subsubsection{SPL\textsuperscript{LIFT}}

\subsection{Modular and Rewritten}

\subsubsection{Feature Transition Systems}

\subsection{Monolithic and Derived}

\subsubsection{TypeChef}

\subsubsection{SuperC}

\subsection{Monolithic and Rewritten}

\subsubsection{Model Transformations}

\section{Discussion}
TODO

\section{Conclusion}
TODO

\bibliographystyle{abbrv}
\bibliography{taxonomy} 
\end{document}
