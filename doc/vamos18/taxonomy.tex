\documentclass[11pt]{article}

%% macros
% reference macros
%\newcommand{\Classen}{\cite{Classen:2010:MCL:1806799.1806850}}
\begin{document}

\title{A Taxonomy of Software Product Line Analyses}
\maketitle

\abstract
TODO

\section{Introduction}
TODO

\section{Motivation}
Compositionality of analyses directly impacts the degree of variability the analysis needs to deal with. Since presence conditions are propositional formulas, propositional reasoning is an essential part of any Product Line analysis. The number of propositional variables (which reflects the degree of variability) in formulas is an important factor when it comes to picking the propositional reasoning component used in the analysis. For example, with hundreds or thousands of variables, SAT solvers are usually used. On the other hand, if the number of variables is much lower (10-20), Binary Decision Digarams (BDDs) tend to perform much better. The difference between both reasoning techniques was demonstrated in the difference in performance between Typechef \cite{} and TODO(the other parser) \cite{}.

\subsection{Compositionaility}

Compositionaility is a property of product-level analyses. An analysis is compositional if it is applied to a component of a system rather than the system as a whole. The granularity of that component varies from one analysis to another. In many programming languages, type checkers work at the level of a compilation unit or a module. Intra-procedural analyses on the other hand work at the level of a single procedure. Non-compositional analyses on the other hand include monolithic model checkers that build a state machine of the whole model with no decomposition into smaller units.

\subsection{Modularity}

In the context of this paper modularity is an attribute of Product Line analyses. By modular, we mean that the Product Line analysis uses the corresponding product-level analysis as a sub-module. This can be done in one of two ways:

A product-level analysis can be used as a black box. In such cases the Product Line analysis calls into the product level analysis and aggregates the results for the whole Product Line.

A product-level analysis can be used as a white box. Some automatic technique is used to lift the implementation of the product-level analysis into a Produt Line analysis. Since this process is automated, we still consider this approach modular.

Non-modular Product Line analyses on the other hand are written from scratch independently of the corresponding product level analysis.

\section{Taxonomy}

\subsection{Compositional and Modular}

\subsection{Compositional and Non-Modular}

\subsection{Non-Compositional and Modular}

\subsection{Non-Compositional and Non-Modular}

\section{Discussion}
TODO

\section{Conclusion}
TODO

%\bibliographystyle{ieeetr}
%\bibliography{taxonomy} 
\end{document}
