\documentclass[12pt]{article}
\begin{document}
\section{Evaluation}

\subsection{Evaluation Metrics}
\subsubsection{Correctness}
Correctness validation would involve enumerating outputs for all products and comparing them with the product-by-product output. Practically we can only do this for relatively small case studies with a few features. Simple FTS examples might be enough.

The rest of the paper should provide correctness proofs, so the proofs together with small validation examples should adequately cover correctness. 

\subsubsection{Performance}
With respect to number of features and number of valid products.

We should demonstrate black-box, white-box and gray-box lifting performance.

Performance is evaluated for the baseline (black-box shallow lifting) and any optimizations that might be implemented. Possible optimizations might include:
\begin{itemize}
	\item Using the feature model to eliminate infeasible feature combinations early on (more computational load for the SAT solver though)
	\item Using Binary Decision Diagrams (BDDs) instead of SAT solvers
	\item Deep-lifting of the List data structure for space optimization and data redundancy elimination
	\item Deep-lifting of other data structures
	\item Deep-lifting of algorithms (as needed, based on profiling results)
\end{itemize}

\subsection{Benchmarks}

\subsubsection{Simple LTS Algorithms}
\begin{itemize}
	\item State reachability
	\item State reachability with witness
\end{itemize}

\subsubsection{Parsing C Code}
\begin{itemize}
	\item GNU CoreUtils
%	\item Case studies from SPLVerifier (contributed to SVComp)
%	\begin{itemize}
%		\item email client
%		\item elevator
%		\item minepump
%	\end{itemize}
	
%	\item \emph{Stretch goal:} x86 Linux kernel files (about 7600 files) used in the evaluation of Typechef and SuperC
\end{itemize}

\subsubsection{C Code Analyses}
On the same case studies used in C code parsing.
\begin{itemize}
	\item Typechecking: We need a C type checker written in a functional language though to lift.
	\item A simple workflow of LLVM analyses (gray-box)
\end{itemize}

%\subsubsection{Model Transformations (graph rules)}
%ICSE'14 paper

\end{document}
